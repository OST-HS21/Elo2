\section{Spannungs-Referenzen und lineare Spannungs-Regler}
\subsection{Einfachste Referenzspannungsquellen}
\subsubsection{Spannungsteiler}
\begin{minipage}[t]{0.3\textwidth}
	\vspace{0pt}								% Abbildung hier einfügen
	\includegraphics[width=0.4\textwidth]{"Spannungsteiler"}
\end{minipage}\hspace{0.05\textwidth}
\begin{minipage}[t]{0.3\textwidth}
	\vspace{0pt}
	\formula{$V_{\text{ref}} = \dfrac{R_2}{R_1 + R_2} \cdot V_{\text{pos}}$}\\
	\formula{$\Delta V_{\text{ref}} = \dfrac{R_2}{R_1+R_2} \cdot \Delta V_{\text{pos}}$}
\end{minipage}
\begin{minipage}[t]{0.3\textwidth}
	\vspace{0pt}
	Bessere Temperaturabhängigkeit
\end{minipage}
\vspace{2mm}

\subsubsection{Diodenreferenz}
\begin{minipage}[t]{0.3\textwidth}
	\vspace{0pt}								% Abbildung hier einfügen
	\includegraphics[width=0.4\textwidth]{"Diodenreferenz"}
	\includegraphics[width=0.4\textwidth]{"Diodenreferenz4dioden"}
\end{minipage}\hspace{0.05\textwidth}
\begin{minipage}[t]{0.3\textwidth}
	\vspace{0pt}
	\formula{$V_{\text{ref}} = V_D = n \cdot \dfrac{k T}{q} \cdot \ln{\left( \dfrac{I_D}{I_S} \right)}$}\\
	\formula{$I_D = I_S \cdot (e^{\frac{V_D}{n \cdot V_T}} - 1)$}\\
	\formula{$I = \dfrac{V_{\text{pos}} - V_D}{R1} \approx \dfrac{V_{\text{pos}}}{R_1}$}\\
	\formula{$\underset{V_{\text{pos}}}{S} = \dfrac{\Delta V_{\text{ref}}}{\Delta I} \cdot \dfrac{I}{V_{\text{ref}}}$}\\
	\formula{$\underset{I}{S} = \dfrac{1}{\ln{\left(\dfrac{I}{I_S} \right)}} \ll 1$}\\
	\formula{$r_D = \dfrac{n \cdot V_T}{I_D}$}
\end{minipage}
\begin{minipage}[t]{0.3\textwidth}
	\vspace{0pt}
	Kleine Speisespannungsabhängigkeit
	Diode hat Temperaturabhängigkeit von $-2 mV/K$
	
	\unitText{$n$}{Parallele Dioden}
\end{minipage}
\vspace{2mm}


\subsubsection{Zenerdioden Referenz}
\begin{minipage}[t]{0.3\textwidth}
	\vspace{0pt}								% Abbildung hier einfügen
	\includegraphics[width=0.4\textwidth]{"ZenerdiodeReferenz"}
\end{minipage}\hspace{0.05\textwidth}
\begin{minipage}[t]{0.3\textwidth}
	\vspace{0pt}
	\formula{$I_{\text{out-max}} = \dfrac{V_{\text{pos}} - V_{\text{ref}}}{R_1}$}
\end{minipage}
\begin{minipage}[t]{0.3\textwidth}
	\vspace{0pt}
	
\end{minipage}
\vspace{2mm}


\subsection{Bootstrap-Referenz}
\begin{minipage}[t]{0.3\textwidth}
	\vspace{0pt}								% Abbildung hier einfügen
	\includegraphics[width=\textwidth]{"BootstrapReferenz"}
\end{minipage}\hspace{0.05\textwidth}
\begin{minipage}[t]{0.3\textwidth}
	\vspace{0pt}
	\formula{$V_{D_1} = n V_T \ln{\left( \dfrac{I_1}{I_S} \right) } = I_2 R_2 = V_{R_2}$}\\
	
\end{minipage}
\begin{minipage}[t]{0.3\textwidth}
	\vspace{0pt}
	\includegraphics[width=\textwidth]{"BootstrapGraph"}
\end{minipage}
\vspace{2mm}


\subsection{Shunt Voltage Referenz}
\begin{minipage}[t]{0.3\textwidth}
	\vspace{0pt}
	\includegraphics[width=\textwidth]{"shuntvoltagereference"}
\end{minipage}\hspace{0.05\textwidth}
\begin{minipage}[t]{0.3\textwidth}
	\vspace{0pt}
	\includegraphics[width=\textwidth]{"shuntvoltagereferenceadjustable"}
\end{minipage}
\begin{minipage}[t]{0.3\textwidth}
	\vspace{0pt}
	
\end{minipage}
\vspace{2mm}


\subsection{Lineare Spannungsregler}
\subsection{Spannungsstabilisierung mit Zenerdiode und Bipolar-Transistor}
\begin{minipage}[t]{0.3\textwidth}
	\vspace{0pt}								% Abbildung hier einfügen
	\includegraphics[width=\textwidth]{"zdiodebiptransistorspann"}
\end{minipage}\hspace{0.05\textwidth}
\begin{minipage}[t]{0.65\textwidth}
	\vspace{0pt}								% Beschreibung und Formeln hier einfügen
	Ausgangsspannung: \formula{$V_{OUT} = V_Z - V_{BE}$}
	
	Die Ausgangsspannung sinkt um ca. 20mV bei Verdoppelung des Strom sowie -2mV/K.
	
\end{minipage}
\vspace{2mm}


\subsection{Linearer Spannungsregler}
\begin{minipage}[t]{0.3\textwidth}
	\vspace{0pt}								% Abbildung hier einfügen
	\includegraphics[width=\textwidth]{"linearerspannungsregler"}
\end{minipage}\hspace{0.05\textwidth}
\begin{minipage}[t]{0.65\textwidth}
	\vspace{0pt}								% Beschreibung und Formeln hier einfügen
	\formula{$U_a = 1+ \dfrac{R_1}{R_2} \cdot U_{ref}$}
\end{minipage}
\vspace{2mm}


\subsection{Low Dropout Regler}
\begin{minipage}[t]{0.3\textwidth}
	\vspace{0pt}								% Abbildung hier einfügen
	\includegraphics[width=\textwidth]{"lowdropoutregler"}
\end{minipage}\hspace{0.05\textwidth}
\begin{minipage}[t]{0.65\textwidth}
	\vspace{0pt}								% Beschreibung und Formeln hier einfügen
	
\end{minipage}
\vspace{2mm}


\subsection{Einstellbare Seriespannungsquelle}
\begin{minipage}[t]{0.3\textwidth}
	\vspace{0pt}								% Abbildung hier einfügen
	\includegraphics[width=\textwidth]{"eintseriespannungsquelle"}
\end{minipage}\hspace{0.05\textwidth}
\begin{minipage}[t]{0.65\textwidth}
	\vspace{0pt}								% Beschreibung und Formeln hier einfügen
	
\end{minipage}
\vspace{2mm}