\section{Feldeffekt-Transistoren}
\subsection{JFET}
\begin{minipage}[t]{0.3\textwidth}
	\vspace{0pt}								% Abbildung hier einfügen
	\includegraphics[width=\textwidth]{"JFET"}
\end{minipage}\hspace{0.05\textwidth}
\begin{minipage}[t]{0.65\textwidth}
	\vspace{0pt}								% Beschreibung und Formeln hier einfügen
	Der Junction-FET oder Sperrschicht-FET enthält eine sperrende Schicht, die entweder als in
	Sperrrichtung betriebener pn- oder Schottky-Übergang aufgebaut ist.
	Ab einer Schwellspannung VTH (für Threshold, Schwelle) nimmt die RLZ den ganzen Kanal ein:
	Der JFET sperrt. Bei hohen VDS enthält der Kanal nahe des Drain's gar keine freien Elektronen,
	d.h. er ist dort „abgeschnürt“, d.h. sehr hochohmig. Der JFET wird zur gesteuerten
	Stromquelle. \refskript{1-3}\\
\end{minipage}
\vspace{2mm}

\begin{minipage}[t]{0.3\textwidth}
	\vspace{0pt}								% Abbildung hier einfügen
	\includegraphics[width=\textwidth]{"JFETkennlinie"}
\end{minipage}\hspace{0.05\textwidth}
\begin{minipage}[t]{0.65\textwidth}
	\vspace{0pt}								% Beschreibung und Formeln hier einfügen
	Ab der sogenannten Pinch-Off-Spannung VDSP (P steht für Pinch-Off, Abschnürung) verläuft die
	Kurve flach: Der JFET wird zur gesteuerten Stromquelle.\\
	
	Doppelte Verstärkung benötigt 4-fachen Strom!
\end{minipage}
\vspace{2mm}



\subsection{MOSFET}
\begin{minipage}[t]{0.3\textwidth}
	\vspace{0pt}								% Abbildung hier einfügen
	\includegraphics[width=\textwidth]{"MOSFET"}
\end{minipage}\hspace{0.05\textwidth}
\begin{minipage}[t]{0.65\textwidth}
	\vspace{0pt}								% Beschreibung und Formeln hier einfügen
	MOSFET steht für Metal Oxide Semiconductor FET. Der Bulk-Anschluss wird meist intern mit dem
	Source-Anschluss verbunden. \refskript{1-5}\\
\end{minipage}
\vspace{2mm}

\begin{minipage}[t]{0.3\textwidth}
	\vspace{0pt}								% Abbildung hier einfügen
	\includegraphics[width=\textwidth]{"MOSFETSperrbereich"}
\end{minipage}\hspace{0.05\textwidth}
\begin{minipage}[t]{0.65\textwidth}
	\vspace{0pt}								% Beschreibung und Formeln hier einfügen
	$V_{GS} < V_{TH}$\\
	p-Substrat: Im Kanal unter dem Gate befinden sich Löcher.
	Zwischen Substrat und Drain resp. Source bestehen pn-Übergänge mit hochohmigen
	Raumladungszonen. Es gibt es keine freien Elektronen im Kanal.\\
	\textbf{Der Mosfet sperrt}
	
\end{minipage}
\vspace{2mm}

\begin{minipage}[t]{0.3\textwidth}
	\vspace{0pt}								% Abbildung hier einfügen
	\includegraphics[width=\textwidth]{"MOSFETWiederstandsbereich"}
\end{minipage}\hspace{0.05\textwidth}
\begin{minipage}[t]{0.65\textwidth}
	\vspace{0pt}								% Beschreibung und Formeln hier einfügen
	$V_{GS} > V_{TH}$ (zwischen 0.5V und 3V)\\
	Elektronen werden unter das Gate angezogen. Es entsteht ein leitender Kanal unter dem Gate.
	Je höher die Gate-Spannung, desto tiefer reicht der leitende Kanal. Der Mosfet arbeitet als
	\textbf{spannungsgesteuerter Widerstand} (filt für $V_{DS} < V{GS} - V-{TH}$).\\	
\end{minipage}
\vspace{2mm}

\begin{minipage}[t]{0.3\textwidth}
	\vspace{0pt}								% Abbildung hier einfügen
	\includegraphics[width=\textwidth]{"MOSFETGesättigt"}
\end{minipage}\hspace{0.05\textwidth}
\begin{minipage}[t]{0.65\textwidth}
	\vspace{0pt}								% Beschreibung und Formeln hier einfügen
	Für $V_{DS} > V_{GS} - V_{TH}$ entsteht eine Zone $\Delta L$ ohne freie Ladungsträger.
	Der Kanal wird abgeschnürt: Ladungsträger können die Zone aber dank einer hoher
	Drain-Spannung überwinden. Der Strom nicht weiter zu. Der Mosfet befindet sich im
	Sättigungsbereich.
	Der Mosfet arbeitet als \textbf{spannungsgesteuerte Stromquelle}. 
\end{minipage}
\vspace{2mm}

\begin{minipage}[t]{0.4\textwidth}
	\vspace{0pt}								% Abbildung hier einfügen
	\includegraphics[width=\textwidth]{"MOSFETKennlinie"}
\end{minipage}\hspace{0.05\textwidth}
\begin{minipage}[t]{0.55\textwidth}
	\vspace{0pt}								% Beschreibung und Formeln hier einfügen
	Anlaufbereich:
	\formula{$I_{D \text{-lin}} = \beta \cdot (V_{GS} - V_{TH} - \frac{1}{2} V_{DS})$}
	\formula{$\beta = \dfrac{W}{L} C_{ox} \cdot \mu_n$}\\
	
	\begin{tabular}{ccc}
	\unitText{$W$}{Kanal Breite}&
	\unitText{$L$}{Kanal Länge} &
	\unitText{$C_{ox}$}{Gate-Kapazität}\\
	\end{tabular}

	Sättigungsbereich:
	\formula{$I_{D \text{-Sat}} = \dfrac{\beta}{2} (V_{GS} - V_{TH})^2 (1 + \lambda \cdot V_{DS})$}
	
	Formel für Anlaufbereich mit $\lambda$ (für kontinuierlichen Übergang):
	\formula{$I_{D \text{-lin}} = \beta \cdot (V_{GS} - V_{TH} - \frac{1}{2} V_{DS}) \cdot
			  V_{DS} \cdot (1 + \lambda \cdot V_{DS})$}
\end{minipage}
\vspace{2mm}

\subsubsection{Ersatzschaltung}
\begin{minipage}[t]{0.3\textwidth}
	\vspace{0pt}								% Abbildung hier einfügen
	\includegraphics[width=\textwidth]{"MOSFETGrossignalErsatzschaltung"}
\end{minipage}\hspace{0.05\textwidth}
\begin{minipage}[t]{0.65\textwidth}
	\vspace{0pt}								% Beschreibung und Formeln hier einfügen
	
\end{minipage}
\vspace{2mm}

\begin{minipage}[t]{0.3\textwidth}
	\vspace{0pt}								% Abbildung hier einfügen
	\includegraphics[width=\textwidth]{"MOSFETKleinsignalersatzschaltung"}
\end{minipage}\hspace{0.05\textwidth}
\begin{minipage}[t]{0.65\textwidth}
	\vspace{0pt}								% Beschreibung und Formeln hier einfügen
	Transkonduktanz im Sättigungsgebiet:
	\formula{$S = g_m = \dfrac{\delta I_D}{\delta V_{GS}} = \beta \cdot (V_{GS} - V_{TH})$}
	\formula{$\beta = KP \dfrac{W}{L}$}
	\formula{$KP = C_{ox} \cdot \mu_n \approx 100 \mu A / V^2$} Für n-Kanal Transistoren.
	\formula{$g_m = \sqrt{2 \beta I_D} = \sqrt{2 \cdot KP \dfrac{W}{L} I_D}$}
	
	$r_{DS}$ ist in erster Näherung proportional zur Gate-Länge $L$.
	\formula{$\dfrac{1}{r_{DS}} = g_{DS} = \dfrac{\delta I_D}{\delta V_{GS}} =
		      \lambda \cdot I_D .\propto \dfrac{1}{L}$}
\end{minipage}
\vspace{2mm}




