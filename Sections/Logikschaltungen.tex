\section{Logikschaltungen}
\subsection{Transistoren in Logikschaltungen}
\subsubsection{Emitter- und Source-Schaltung}
\begin{minipage}[t]{0.3\textwidth}
	\vspace{0pt}								% Abbildung hier einfügen
	\includegraphics[width=\textwidth]{"EmitterSourceSchaltung"}
\end{minipage}\hspace{0.05\textwidth}
\begin{minipage}[t]{0.65\textwidth}
	\vspace{0pt}								% Beschreibung und Formeln hier einfügen
	
\end{minipage}
\vspace{2mm}

%--------------------------------------------------------------------------------

\subsubsection{CMOS Inverter}
\begin{minipage}[t]{0.3\textwidth}
	\vspace{0pt}								% Abbildung hier einfügen
	\includegraphics[width=0.4\textwidth]{"CMOSInverter"}
	\includegraphics[width=0.4\textwidth]{"CMOSInverterKennlinie"}
\end{minipage}\hspace{0.05\textwidth}
\begin{minipage}[t]{0.65\textwidth}
	\vspace{0pt}								% Beschreibung und Formeln hier einfügen
	Die grössten Verluste entstehen beim Schalten (bei NMOS/PMOS sat).
\end{minipage}
\vspace{2mm}

%--------------------------------------------------------------------------------

\subsubsection{CMOS Gatter}
\begin{minipage}[t]{0.45\textwidth}
	\vspace{0pt}								% Abbildung hier einfügen
	\subsubsection{NMOS}
	\includegraphics[width=\textwidth]{"NMOSLogik"}
\end{minipage}\hspace{0.05\textwidth}
\begin{minipage}[t]{0.45\textwidth}
	\vspace{0pt}								% Beschreibung und Formeln hier einfügen
	\subsubsection{PMOS}
	\includegraphics[width=\textwidth]{"PMOSLogik"}
\end{minipage}\\
\begin{minipage}[t]{0.45\textwidth}
	\vspace{0pt}								% Abbildung hier einfügen
	\subsubsection{NAND-Gatter}
	\includegraphics[width=\textwidth, height=4cm, keepaspectratio]{"NANDGate"}
	\begin{tabular}{|c|c|c|}
		\hline
		A & B & Out \\
		\hline
		0 & 0 & 1 \\
		\hline
		0 & 1 & 1 \\
		\hline
		1 & 0 & 1 \\
		\hline
		1 & 1 & 0 \\
		\hline
	\end{tabular}
\end{minipage}\hspace{0.05\textwidth}
\begin{minipage}[t]{0.45\textwidth}
	\vspace{0pt}								% Beschreibung und Formeln hier einfügen
	\subsubsection{NOR-Gate}
	\includegraphics[width=\textwidth, height=4cm, keepaspectratio]{"NORGate"}
	\begin{tabular}{|c|c|c|}
		\hline
		A & B & Out \\
		\hline
		0 & 0 & 1 \\
		\hline
		0 & 1 & 0 \\
		\hline
		1 & 0 & 0 \\
		\hline
		1 & 1 & 0 \\
		\hline
	\end{tabular}
\end{minipage}
\vspace{2mm}

%################################################################################

\subsection{Logik-Pegel: CMOS vs TTL}
\includegraphics[width=0.2\textwidth]{"LogikPegel"}
\includegraphics[width=0.3\textwidth]{"LogikPegel2"}


\subsubsection{Schmitt-Trigger}
\begin{minipage}[t]{0.3\textwidth}
	\vspace{0pt}
	\includegraphics[width=0.6\textwidth]{"Schmitttgrigger"}
\end{minipage}\hspace{0.05\textwidth}
\begin{minipage}[t]{0.3\textwidth}
	\vspace{0pt}
	\includegraphics[width=0.6\textwidth]{"SchmitttgriggerGraph"}
\end{minipage}\\
\begin{minipage}[t]{0.45\textwidth}
	\vspace{0pt}								% Abbildung hier einfügen
	\subsubsection{Nichtinvertierend}
	\includegraphics[width=0.6\textwidth]{"SchmittTriggerNichtinvertierend"}
\end{minipage}\hspace{0.05\textwidth}
\begin{minipage}[t]{0.45\textwidth}
	\vspace{0pt}								% Beschreibung und Formeln hier einfügen
	\subsubsection{Invertierend}
	\includegraphics[width=0.6\textwidth]{"SchmittTriggerInvertierend"}
\end{minipage}
\vspace{2mm}

%################################################################################

\subsection{CMOS Verzögerung}
\begin{minipage}[t]{0.45\textwidth}
	\vspace{0pt}								% Abbildung hier einfügen
	\subsubsection{MOSFET als Widerstand}
	\includegraphics[width=0.6\textwidth]{"MOSFETWiderstand"}
\end{minipage}\hspace{0.05\textwidth}
\begin{minipage}[t]{0.45\textwidth}
	\vspace{0pt}								% Beschreibung und Formeln hier einfügen
	\subsubsection{MOSFET als Stromquelle}
	\includegraphics[width=0.6\textwidth]{"MOSFETStromquelle"}
\end{minipage}
\vspace{2mm}

%################################################################################

\subsection{Alternative Logik-Familien}
\subsubsection{Dioden Logik}
\begin{minipage}[t]{0.34\textwidth}
	\vspace{0pt}
	\textbf{AND Gate}\\
	\includegraphics[width=0.6\textwidth]{"DiodenAND"}
\end{minipage}
\begin{minipage}[t]{0.34\textwidth}
	\vspace{0pt}
	\textbf{OR Gate}\\
	\includegraphics[width=0.6\textwidth]{"DiodenOR"}
\end{minipage}
\vspace{2mm}

%--------------------------------------------------------------------------------

\begin{minipage}[t]{0.45\textwidth}
	\vspace{0pt}								% Abbildung hier einfügen
	\subsubsection{RTL: Resistor-Transistor-Logik}
	\textbf{NOR Gate}\\
	\includegraphics[width=0.6\textwidth]{"RTL"}
\end{minipage}\hspace{0.05\textwidth}
\begin{minipage}[t]{0.45\textwidth}
	\vspace{0pt}								% Beschreibung und Formeln hier einfügen
	\subsubsection{DTL: DiodenTransistor-Logik}
	\textbf{AND Gate}\\
	\includegraphics[width=0.6\textwidth]{"DTL"}
\end{minipage}
\vspace{2mm}

%################################################################################

\subsection{High Speed Logik}
\subsubsection{Emitter Coupled Logic ECL}
\begin{minipage}[t]{0.45\textwidth}
	\vspace{0pt}								% Abbildung hier einfügen
	\includegraphics[width=\textwidth]{"ECL"}
\end{minipage}\hspace{0.05\textwidth}
\begin{minipage}[t]{0.45\textwidth}
	\vspace{0pt}								% Beschreibung und Formeln hier einfügen
	Die Eingangsspannung $V_I$ wird mit der fixen Referenzspannung $V_R$ verglichen.
\end{minipage}
\vspace{2mm}

%--------------------------------------------------------------------------------

\subsubsection{Differenzverstärker}
\begin{minipage}[t]{0.3\textwidth}
	\vspace{0pt}								% Abbildung hier einfügen
	\includegraphics[width=\textwidth]{"Differenzverstärker"}
\end{minipage}\hspace{0.05\textwidth}
\begin{minipage}[t]{0.65\textwidth}
	\vspace{0pt}								% Beschreibung und Formeln hier einfügen
	
\end{minipage}
\vspace{2mm}

%--------------------------------------------------------------------------------

\subsubsection{Current mode logic CML}
\begin{minipage}[t]{0.3\textwidth}
	\vspace{0pt}								% Abbildung hier einfügen
	\includegraphics[width=\textwidth]{"CurrentModeLogic"}
\end{minipage}\hspace{0.05\textwidth}
\begin{minipage}[t]{0.65\textwidth}
	\vspace{0pt}								% Beschreibung und Formeln hier einfügen
	Differenz-Spannung wird 800mV (+/-400mV@16mA)
\end{minipage}
\vspace{2mm}

%################################################################################

\subsection{Low Voltage Differential Signaling LVDS}
\subsubsection{LVDS mit Leitung}
\begin{minipage}[t]{0.3\textwidth}
	\vspace{0pt}								% Abbildung hier einfügen
	\includegraphics[width=\textwidth]{"LVDS"}
\end{minipage}\hspace{0.05\textwidth}
\begin{minipage}[t]{0.65\textwidth}
	\vspace{0pt}								% Beschreibung und Formeln hier einfügen
	LVDS muss differentiell abgeschlossen werden mit 100Ohm.
\end{minipage}
\vspace{2mm}

%--------------------------------------------------------------------------------

\subsubsection{Spannungspegel}
\begin{minipage}[t]{0.3\textwidth}
	\vspace{0pt}								% Abbildung hier einfügen
	\includegraphics[width=\textwidth]{"Spannungspegel"}
\end{minipage}\hspace{0.05\textwidth}
\begin{minipage}[t]{0.65\textwidth}
	\vspace{0pt}								% Beschreibung und Formeln hier einfügen
	
\end{minipage}
\vspace{2mm}

%--------------------------------------------------------------------------------

\subsubsection{Datenrate und Übertragungsdistanz}
\begin{minipage}[t]{0.3\textwidth}
	\vspace{0pt}								% Abbildung hier einfügen
	\includegraphics[width=\textwidth]{"DatenratenDistanz"}
\end{minipage}\hspace{0.05\textwidth}
\begin{minipage}[t]{0.65\textwidth}
	\vspace{0pt}
	Signalübertragung über elektrische Leitungen:								% Beschreibung und Formeln hier einfügen
	\begin{itemize}
		\item Koaxkabel
		\item Twisted-Pair-Kabel
		\item HDMI
		\item USB-C
	\end{itemize}
\end{minipage}
\vspace{2mm}




