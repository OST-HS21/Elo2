\section{Logikschaltungen}
\subsection{Transistoren in Logikschaltungen}
\subsubsection{Emitter- und Source-Schaltung}
\begin{minipage}[t]{0.3\textwidth}
	\vspace{0pt}								% Abbildung hier einfügen
	\includegraphics[width=\textwidth]{"EmitterSourceSchaltung"}
\end{minipage}\hspace{0.05\textwidth}
\begin{minipage}[t]{0.65\textwidth}
	\vspace{0pt}								% Beschreibung und Formeln hier einfügen
	
\end{minipage}
\vspace{2mm}

\subsubsection{CMOS Inverter}
\begin{minipage}[t]{0.3\textwidth}
	\vspace{0pt}								% Abbildung hier einfügen
	\includegraphics[width=0.4\textwidth]{"CMOSInverter"}
	\includegraphics[width=0.4\textwidth]{"CMOSInverterKennlinie"}
\end{minipage}\hspace{0.05\textwidth}
\begin{minipage}[t]{0.65\textwidth}
	\vspace{0pt}								% Beschreibung und Formeln hier einfügen
	Die grössten Verluste entstehen beim Schalten (bei NMOS/PMOS sat).
\end{minipage}
\vspace{2mm}


\subsubsection{CMOS Gatter}
\begin{minipage}[t]{0.45\textwidth}
	\vspace{0pt}								% Abbildung hier einfügen
	\subsubsection{NMOS}
	\includegraphics[width=0.6\textwidth]{"NMOSLogik"}
\end{minipage}\hspace{0.05\textwidth}
\begin{minipage}[t]{0.45\textwidth}
	\vspace{0pt}								% Beschreibung und Formeln hier einfügen
	\subsubsection{PMOS}
	\includegraphics[width=0.6\textwidth]{"PMOSLogik"}
\end{minipage}
\vspace{2mm}


\subsubsection{Schmitt-Trigger}
\begin{minipage}[t]{0.45\textwidth}
	\vspace{0pt}								% Abbildung hier einfügen
	\subsubsection{Nichtinvertierend}
	\includegraphics[width=0.6\textwidth]{"SchmittTriggerNichtinvertierend"}
\end{minipage}\hspace{0.05\textwidth}
\begin{minipage}[t]{0.45\textwidth}
	\vspace{0pt}								% Beschreibung und Formeln hier einfügen
	\subsubsection{Invertierend}
	\includegraphics[width=0.6\textwidth]{"SchmittTriggerInvertierend"}
\end{minipage}
\vspace{2mm}


\subsubsection{CMOS Verzögerung}
\begin{minipage}[t]{0.45\textwidth}
	\vspace{0pt}								% Abbildung hier einfügen
	\subsubsection{MOSFET als Widerstand}
	\includegraphics[width=0.6\textwidth]{"MOSFETWiderstand"}
\end{minipage}\hspace{0.05\textwidth}
\begin{minipage}[t]{0.45\textwidth}
	\vspace{0pt}								% Beschreibung und Formeln hier einfügen
	\subsubsection{MOSFET als Stromquelle}
	\includegraphics[width=0.6\textwidth]{"MOSFETStromquelle"}
\end{minipage}
\vspace{2mm}


\subsubsection{Differenzverstärker}
\begin{minipage}[t]{0.3\textwidth}
	\vspace{0pt}								% Abbildung hier einfügen
	\includegraphics[width=\textwidth]{"Differenzverstärker"}
\end{minipage}\hspace{0.05\textwidth}
\begin{minipage}[t]{0.65\textwidth}
	\vspace{0pt}								% Beschreibung und Formeln hier einfügen
	
\end{minipage}
\vspace{2mm}


\subsection{Signalübertragung}
\subsubsection{Emitter Coupled Logic ECL}
\begin{minipage}[t]{0.3\textwidth}
	\vspace{0pt}								% Abbildung hier einfügen
	\includegraphics[width=\textwidth]{"CurrentModeLogic"}
\end{minipage}\hspace{0.05\textwidth}
\begin{minipage}[t]{0.65\textwidth}
	\vspace{0pt}								% Beschreibung und Formeln hier einfügen
	Differenz-Spannung wird 800mV (+/-400mV@16mA)
\end{minipage}
\vspace{2mm}

\subsubsection{Low Voltage Differential Signaling LVDS}
\begin{minipage}[t]{0.3\textwidth}
	\vspace{0pt}								% Abbildung hier einfügen
	\includegraphics[width=\textwidth]{"LVDS"}
\end{minipage}\hspace{0.05\textwidth}
\begin{minipage}[t]{0.65\textwidth}
	\vspace{0pt}								% Beschreibung und Formeln hier einfügen
	LVDS muss differentiell abgeschlossen werden mit 100Ohm.
\end{minipage}
\vspace{2mm}

\subsubsection{Spannungspegel}
\begin{minipage}[t]{0.3\textwidth}
	\vspace{0pt}								% Abbildung hier einfügen
	\includegraphics[width=\textwidth]{"Spannungspegel"}
\end{minipage}\hspace{0.05\textwidth}
\begin{minipage}[t]{0.65\textwidth}
	\vspace{0pt}								% Beschreibung und Formeln hier einfügen
	
\end{minipage}
\vspace{2mm}


\subsubsection{Datenrate und Übertragungsdistanz}
\begin{minipage}[t]{0.3\textwidth}
	\vspace{0pt}								% Abbildung hier einfügen
	\includegraphics[width=\textwidth]{"DatenratenDistanz"}
\end{minipage}\hspace{0.05\textwidth}
\begin{minipage}[t]{0.65\textwidth}
	\vspace{0pt}
	Signalübertragung über elektrische Leitungen:								% Beschreibung und Formeln hier einfügen
	\begin{itemize}
		\item Koaxkabel
		\item Twisted-Pair-Kabel
		\item HDMI
		\item USB-C
	\end{itemize}
\end{minipage}
\vspace{2mm}




